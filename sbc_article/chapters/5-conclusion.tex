\section{Conclusão}

Este trabalho analisou a aplicação prática de ferramentas de IA gerativa na criação de ativos visuais para jogos digitais, destacando sua capacidade de agilizar a prototipagem e aumentar a diversidade estética. A utilização de diferentes modelos de IA, bem como técnicas como Inpainting, ControlNet e LoRA, permitiu maior controle sobre composição, poses, estilos e correções localizadas das imagens. A experimentação demonstrou que usuários sem experiência prévia em modelagem 3D podem transformar conceitos visuais em ativos 3D funcionalmente integráveis a um protótipo de jogo.

Entretanto, a geração automática ainda apresenta limitações significativas. O controle sobre o resultado final é parcial, e os modelos 3D frequentemente requerem ajustes manuais em softwares como Blender para corrigir topologia, geometria irregular ou superfícies incompletas. Assim, o julgamento humano permanece essencial para validar, selecionar e refinar os resultados, garantindo coerência estética e precisão estrutural.

O estudo também abordou aspectos éticos e legais relacionados ao uso de IA gerativa, ressaltando riscos de violação de direitos autorais e mimese de estilos de artistas presentes nos dados de treinamento. Além disso, designers têmem que o conteúdo gerado possa resultar em plágio não intencional ou disputas jurídicas, tornando essencial que estúdios de jogos implementem políticas claras sobre o uso da IA e os limites da autoria criativa \cite{alharthi}. Para uso responsável, recomenda-se conhecer a origem dos modelos e datasets, documentar o processo, manter transparência sobre a participação da IA e adotar práticas que respeitem a diversidade, acompanhando legislações emergentes.

Em síntese, a IA gerativa deve ser entendida como uma ferramenta de apoio à criatividade humana. Sua integração eficiente acelera a produção de conteúdos jogáveis, amplia possibilidades artísticas e democratiza o acesso à modelagem 3D, mas depende de supervisão, reflexão crítica e responsabilidade ética e legal do usuário. O uso pleno e sustentável da IA resulta do equilíbrio entre inovação tecnológica e julgamento humano.