\section{Introdução}

A modelagem \ac{3d} é essencial na criação de jogos digitais, animações e experiências imersivas, mas o desenvolvimento de ativos \ac{3d} exige tempo, habilidades técnicas e conhecimento em arte digital. Com o crescente interesse de desenvolvedores e entusiastas sem formação em computação gráfica, surge a necessidade de ferramentas que facilitem o processo criativo e acelerem a produção de conteúdo. 

Recentemente, modelos de \ac{ia} gerativa, como o Stable Diffusion, popularizaram-se por permitir a geração de imagens e texturas a partir de descrições textuais ou imagens de referência, oferecendo recursos como inpainting, upscaling, ControlNet e \ac{lora}. Essas técnicas ajudaram a popularizar a \ac{ia} gerativa, tornando-a poderosa para design conceitual em qualquer disciplina que exija criatividade visual, incluindo estágios iniciais de projetos arquitetônicos, como ideação, esboço e modelagem \cite{ploennigs2023}. As capacidades gerativas dessas ferramentas provavelmente alteraram fundamentalmente os processos criativos, influenciando a forma como ideias são formuladas e transformadas em produtos \cite{epstein2023}.

Apesar de permitir prototipagem rápida e acelerar o design, a \ac{ia} gerativa apresenta curva de aprendizado significativa, exigindo compreensão de parâmetros e entradas para gerar resultados desejados. Diante disso, este trabalho propõe a aplicação dessas ferramentas na modelagem \ac{3d} para jogos digitais, com a elaboração de um manual prático que sistematiza fluxos de trabalho com Stable Diffusion e suas interfaces Automatic1111 e ComfyUI, e a criação de um protótipo de jogo na Unreal Engine 5 que demonstra a integração de ativos e personagens gerados por \ac{ia}. Os resultados indicam que tais ferramentas facilitam a modelagem \ac{3d} para iniciantes, apoiam o processo criativo e promovem eficiência, democratização do acesso e inovação na criação de conteúdo digital.

\subsection{Objetivos}

O objetivo deste trabalho é investigar o uso de ferramentas de \ac{ia} gerativa na criação de imagens e modelos \ac{3d} para apoiar o desenvolvimento visual de jogos digitais. Busca-se demonstrar como essas ferramentas podem agilizar a concepção de personagens, objetos e cenários, além de gerar modelos \ac{3d} a partir de imagens, consolidando os procedimentos em um manual prático voltado a iniciantes. 

Os objetivos específicos incluem explorar conceitos de \ac{ia} gerativa, investigar interfaces como Automatic1111 e ComfyUI, aplicar técnicas de Image-to-\ac{3d}, desenvolver o manual prático, avaliar o impacto no processo criativo e criar um protótipo de jogo utilizando os ativos gerados.
