% resumo em inglês
\begin{abstract}
	\begin{otherlanguage*}{english}
		This work investigates the use of Generative Artificial Intelligence, particularly the Stable Diffusion model, applied to 3D modeling and its integration into digital game development. The objective was to develop a practical guide that demonstrates how these techniques can support the generation of ideas, visual concepts, meshes, and textures for three-dimensional models. The methodology involved hands-on experimentation with different workflows using Stable Diffusion and auxiliary tools, evaluating the quality of the results and their applicability in the 3D creation process. As a result, a practical manual was produced to guide beginners in the agile creation of 3D assets, along with the development of a game prototype in Unreal Engine 5 using AI-generated elements. The results indicate that generative AI can significantly reduce production time and improve accessibility to modeling processes, especially for novice users. The contributions of this study include the systematization of an accessible workflow and the demonstration of AI’s potential as a support tool for 3D modeling in games.
		
		\textbf{Keywords}: Generative Artificial Intelligence. Stable Diffusion. 3D Modeling. Game Engine.
	\end{otherlanguage*}
\end{abstract}

\begin{resumo}
    Este trabalho investiga o uso de \ac{ia} Gerativa, em especial o modelo Stable Diffusion, aplicada à modelagem \ac{3d} e sua integração no desenvolvimento de jogos digitais. O objetivo foi elaborar um manual prático que demonstra como essas técnicas podem apoiar a geração de ideias, conceitos visuais, malhas e texturas para modelos tridimensionais. A metodologia envolveu experimentação prática com diferentes fluxos de trabalho utilizando Stable Diffusion e ferramentas auxiliares, avaliando a qualidade dos resultados e sua aplicabilidade no processo de criação \ac{3d}. Como resultado, foi produzido um manual prático que orienta iniciantes na criação ágil de ativos \ac{3d}, além da construção de um protótipo de jogo na Unreal Engine 5 utilizando elementos gerados por \ac{ia}. Os resultados indicam que a \ac{ia} gerativa pode reduzir significativamente o tempo de produção e facilitar o acesso a processos de modelagem, especialmente para usuários iniciantes. As contribuições deste estudo incluem a sistematização de um fluxo de trabalho acessível e a demonstração do potencial da \ac{ia} como ferramenta de apoio à modelagem \ac{3d} em jogos.

    \textbf{Palavras-chave}: Inteligência Artificial Gerativa. Stable Diffusion. Modelagem 3D. Game Engine.
\end{resumo}