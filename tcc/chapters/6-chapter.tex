% ----------------------------------------------------------
\chapter{Questões Éticas e Cuidados no Uso de IA Gerativa}
% ----------------------------------------------------------

A popularização das ferramentas de IA gerativa proporcionou maior praticidade, facilidade e produtividade para diversas áreas criativas e técnicas. Por exemplo, um estudo com desenvolvedores de jogos indicou que mais de 75\% dos participantes afirmaram que essas ferramentas ajudam a concluir tarefas com mais rapidez, aumentam a produtividade e melhoram a qualidade do design \cite{alharthi}. No entanto, também gerou preocupações relacionadas à ética, autoria, originalidade e uso responsável dessas tecnologias. Este capítulo aborda os principais desafios e cuidados necessários para garantir o uso ético e consciente da IA na produção de conteúdos digitais.

Durante o treinamento, os modelos de IA são alimentados com grandes volumes de dados, que podem incluir obras protegidas por direitos autorais, como ilustrações, fotografias, esculturas digitais ou animações. Caso essas obras sejam utilizadas sem autorização dos autores, pode ocorrer violação de direitos autorais, mesmo que o resultado final gerado pela IA seja diferente da obra original. Os designers temem que o uso de conteúdo gerado por IA possa resultar em plágio não intencional ou disputas jurídicas, tornando essencial que os estúdios de jogos implementem políticas claras sobre como a IA gerativa é utilizada e onde se estabelecem os limites da autoria criativa \cite{alharthi}. 

Um caso particular de preocupação envolve a mimese de estilo, fenômeno pelo qual modelos de IA conseguem reproduzir características estilísticas de artistas presentes no conjunto de treinamento, como cores, texturas, pinceladas e composições. Nesse processo, a IA não copia obras específicas, mas gera novas imagens que parecem ter sido criadas por um determinado artista, mesmo sem seu consentimento. Essa capacidade de imitar estilos levanta questões relevantes sobre autoria, originalidade e reconhecimento na produção artística digital \cite{sarcevic}.

Além das questões relacionadas ao treinamento dos modelos, surge também a discussão sobre a autoria e a propriedade dos conteúdos gerados por IA. Atualmente, não há um consenso claro sobre quem deve possuir esses conteúdos, nem sobre como os direitos devem ser distribuídos entre criadores, desenvolvedores de modelos e usuários finais.

No contexto de desenvolvimento de jogos, isso significa que equipes que utilizam IA gerativa para criar personagens, cenários ou texturas devem estar cientes de que, embora o conteúdo seja original em sua forma final, elementos do treinamento podem ter origem em obras de terceiros. Para garantir o uso ético, legal e responsável de ferramentas de IA gerativa, é fundamental que os usuários conheçam a origem dos modelos e datasets utilizados, privilegiando conteúdos com licenças adequadas ou de domínio público e evitando a utilização de obras protegidas sem autorização. Deve-se manter transparência quanto à participação da IA na criação de ativos, informando quando imagens ou modelos foram gerados com auxílio de IA. A documentação do processo de criação, incluindo \textit{workflows}, parâmetros e imagens de entrada, é essencial para comprovar a originalidade e facilitar a gestão de possíveis questionamentos legais. Por fim, os usuários devem considerar os impactos sociais e culturais de suas criações, adotando práticas que respeitem a diversidade e mantendo-se atualizados quanto às legislações e diretrizes emergentes sobre o uso de IA gerativa.