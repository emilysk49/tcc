% ----------------------------------------------------------
\chapter{Resultados e Limitações do Uso de IA gerativa }
% ----------------------------------------------------------

A aplicação de ferramentas de IA gerativa no processo de criação de conteúdos visuais trouxe resultados expressivos em termos de agilidade, variedade e experimentação criativa. Este capítulo apresenta uma análise dos principais pontos fortes e limitações observados durante o desenvolvimento do manual prático e na geração de modelos 3D utilizados no jogo. 

% ----------------------------------------------------------
\section{Pontos Fortes Observados}
% ----------------------------------------------------------

Durante a experimentação prática, observou-se que as ferramentas de IA gerativa apresentaram diversos benefícios que contribuíram para o processo criativo na produção de ativos visuais. Entre os principais pontos fortes, destaca-se a agilidade na geração de ideias e conceitos, permitindo ao usuário explorar rapidamente diferentes possibilidades estéticas e estilísticas. Esse ganho de velocidade reduziu significativamente o tempo necessário para criar e testar propostas visuais, especialmente nas fases iniciais de concepção de personagens.

Outro aspecto relevante foi a diversidade e riqueza dos resultados obtidos. O uso de diferentes modelos de difusão, como Realistic Vision, DreamShaper e RevAnimated, proporcionou uma ampla variação de estilos e níveis de realismo, ampliando as possibilidades de experimentação artística. A integração de modelos LoRA permitiu combinar estilos específicos, personalizando ainda mais os resultados. O ControlNet se mostrou uma ferramenta valiosa para aumentar o controle sobre a composição, poses e contornos das imagens, enquanto os recursos de inpainting possibilitaram ajustes precisos e correções localizadas, garantindo maior fidelidade e coerência visual. Essa característica contribuiu para uma abordagem mais exploratória e iterativa do processo criativo, na qual o usuário pode avaliar, comparar e refinar ideias de forma dinâmica.

Além da geração de imagens 2D, o manual também explorou o uso de IA gerativa para criação de modelos 3D. Essa abordagem demonstrou grande potencial na geração rápida de malhas e texturas, permitindo que usuários transformassem conceitos visuais em ativos tridimensionais de maneira eficiente. Entre os principais pontos positivos, destaca-se a agilidade na prototipagem de modelos 3D, que reduz significativamente o tempo necessário em comparação com a modelagem manual tradicional. Além disso, essa ferramenta possibilita que usuários sem conhecimento prévio em modelagem 3D possam gerar personagens, cenários e objetos apenas a partir de ideias ou referências visuais, democratizando o acesso à criação de conteúdos tridimensionais.

% ----------------------------------------------------------
\section{Limitações}
% ----------------------------------------------------------

Apesar dos avanços e benefícios identificados, o uso de IA gerativa também apresentou limitações técnicas e conceituais que precisam ser consideradas para um uso eficaz dessas tecnologias. Uma das principais dificuldades foi o controle limitado sobre os resultados. Mesmo com o uso de prompts detalhados e ajustes de parâmetros, as imagens geradas nem sempre refletiam com precisão a intenção do usuário, apresentando variações imprevisíveis. 

A geração de modelos 3D por IA ainda apresenta limitações importantes. Modelos como Hunyuan 3D 2.0 conseguem criar malhas e texturas de forma rápida, mas muitas vezes os resultados exigem ajustes manuais para alcançar nível de qualidade profissional, principalmente em detalhes complexos ou proporções precisas. A geometria gerada pode apresentar irregularidades, falhas de topologia, superfícies incompletas ou artefatos indesejados, tornando necessário o uso de softwares de modelagem tradicionais para refinamento. Esses desafios indicam que, embora a IA seja uma ferramenta poderosa de prototipagem, a intervenção humana continua sendo essencial para garantir precisão e qualidade final.

Portanto, apesar de sua eficiência na fase exploratória, a IA gerativa ainda depende fortemente do julgamento humano. O papel do criador continua essencial para selecionar, ajustar e validar os resultados, assegurando coerência estética, intencionalidade e autenticidade no produto final.

A análise dos resultados evidencia que as ferramentas de IA gerativa possuem grande potencial como aliadas na produção de conteúdos digitais, especialmente por sua capacidade de acelerar processos e estimular a experimentação visual. Contudo, suas limitações reforçam a necessidade de uma abordagem crítica e consciente, em que o usuário compreenda o funcionamento da tecnologia, suas implicações éticas e seus limites criativos.