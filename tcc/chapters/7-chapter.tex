% ----------------------------------------------------------
\chapter{Conclusão}
% ----------------------------------------------------------

O desenvolvimento do manual prático permitiu analisar de forma sistemática a aplicação de ferramentas de IA gerativa na criação de conteúdos visuais e modelos 3D para jogos. Através da experimentação com plataformas como Automatic1111 e ComfyUI, foi possível demonstrar como essas tecnologias agilizam a produção, aumentam a diversidade estética e permitem que usuários, mesmo sem experiência prévia em modelagem, transformem ideias em modelos 3D concretos de forma rápida e iterativa. Além disso, a documentação detalhada do processo, incluindo \textit{workflows}, parâmetros de geração e resultados comparativos, contribuiu para a criação de um material educativo que pode servir como referência prática para futuros desenvolvedores e estudantes de computação gráfica e design de jogos.

O manual também evidenciou limitações técnicas, como controle ainda parcial sobre os resultados gerados e a necessidade de ajustes manuais em modelos 3D. Além disso, a experiência prática ressaltou a importância de conhecer a origem dos modelos e datasets, manter transparência quanto à participação da IA e adotar práticas responsáveis e conscientes durante a criação através da ferramenta de IA gerativa. Portanto, reforça-se que a IA gerativa deve ser encarada como uma ferramenta de apoio e não como substituta do processo criativo humano.

Por fim, a integração entre criatividade humana e IA gerativa se mostra capaz de acelerar a produção de conteúdos digitais, democratizar o acesso à criação de elementos visuais e impulsionar a experimentação artística. Entretanto, essa integração requer supervisão constante, reflexão crítica sobre os resultados gerados e responsabilidade ética e legal por parte dos usuários, especialmente considerando aspectos relacionados à autoria, originalidade e licenciamento de conteúdos digitais. O trabalho evidencia que, embora a IA gerativa represente um avanço significativo na criação digital, seu uso pleno e sustentável depende do equilíbrio entre inovação tecnológica e julgamento crítico humano, abrindo caminhos para futuras pesquisas e aplicações em jogos, educação e indústria criativa.